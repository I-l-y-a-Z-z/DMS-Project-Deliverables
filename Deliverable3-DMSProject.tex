\documentclass[a4paper,12pt]{article}
\usepackage[utf8]{inputenc}
\usepackage{geometry}
\usepackage{graphicx}   % images
\usepackage{fancyhdr}   % headers/footers
\usepackage{tcolorbox}
\usepackage{listings}
\usepackage{xcolor}
\usepackage{float}
\usepackage{mdframed}
\usepackage{enumitem}
\usepackage{amsmath}
\usepackage{amssymb}

\geometry{margin=1in}

% ---------- LaTeX settings for SQL code ----------
\definecolor{codegreen}{rgb}{0,0.6,0}
\definecolor{codegray}{rgb}{0.5,0.5,0.5}
\definecolor{codepurple}{rgb}{0.58,0,0.82}
\definecolor{backcolour}{rgb}{0.95,0.95,0.92}

\lstset{
  language=SQL,
  basicstyle=\ttfamily\small,
  keywordstyle=\color{blue}\bfseries,
  stringstyle=\color{brown},
  commentstyle=\color{gray}\itshape,
  showstringspaces=false,
  frame=single,
  breaklines=true,
  captionpos=b
}

% ---------- Header/Footer settings ----------
\setlength{\headheight}{36pt}
\setlength{\headsep}{18pt}
\renewcommand{\headrulewidth}{0.4pt}
\fancyhf{}
\fancyhead[L]{\includegraphics[width=0.13\textwidth, keepaspectratio]{Figures/UM6Plogo.png}}
\fancyhead[R]{\includegraphics[width=0.13\textwidth, keepaspectratio]{Figures/CC.jpg}}
\fancyfoot[L]{Data Management Lab}
\fancyfoot[R]{Prof. Karima Echihabi}
\fancyfoot[C]{Page \thepage}

\begin{document}
% ---------- Title Page (as provided) ----------
\thispagestyle{empty}
\begin{center}
  \includegraphics[width=0.25\textwidth]{Figures/UM6Plogo.png}\hfill
  \includegraphics[width=0.25\textwidth]{Figures/CC.jpg}
  \vspace{1.2cm}

  {\LARGE \textbf{Deliverable 3: Relational Algebra, SQL and Functional Dependencies}}\\[0.6cm]
  {\large \textbf{Data Management Course}}\\[0.2cm]
  {\large UM6P College of Computing}\\[0.8cm]

  {\normalsize \textbf{Professor:} Karima Echihabi \quad 
   \textbf{Program:} Computer Engineering}\\[0.1cm]
  {\normalsize \textbf{Session:} Fall 2025}\\[1cm]

  \rule{0.9\textwidth}{0.5pt}\\[0.5cm]
  {\large \textbf{Team Information}} \\[0.3cm]
  \begin{tabular}{|l|l|}
    \hline
    \textbf{Team Name} & alfari9 alkhari9 \\ \hline
    \textbf{Member 1}  & Ilyas Rahmouni \\ \hline
    \textbf{Member 2}  & Malak Koulat   \\ \hline
    \textbf{Member 3}  & Aymane Raiss   \\ \hline
    \textbf{Member 4}  & Zakaria Harira   \\ \hline
    \textbf{Member 5}  & Youness Latif   \\ \hline
    \textbf{Member 6}  & Rayane Khaldi   \\ \hline
    \textbf{Member 7}  & Younes Lougnidi   \\ \hline
    \textbf{Repository Link} & \texttt{https://github.com/...} \\ \hline
  \end{tabular}
  \rule{0.9\textwidth}{0.5pt}\\
\end{center}
\clearpage
\pagestyle{fancy}

% ---------- Sections for Lab 4 ----------
\section{Part 1: Relational Algebra and SQL}

\subsection*{1. Find the names of patients who have had at least one clinical activity handled by active staff.}
\noindent\textbf{Relational Algebra:}\\
\begin{align*}
\pi_{\text{FullName}} \Big( 
    & \sigma_{\text{Status='Active'}} (\text{Staff}) \\
    & \Join_{\text{Staff.STAFF\_ID} = \text{ClinicalActivity.STAFF\_ID}} \text{ClinicalActivity} \\
    & \Join_{\text{ClinicalActivity.IID} = \text{Patient.IID}} \text{Patient}
\Big)
\end{align*}
\noindent\textbf{SQL:}\\
\begin{lstlisting}[caption={Query for patients treated by active staff.}]
SELECT P.FullName
FROM Patient P
JOIN ClinicalActivity CA ON P.IID = CA.IID
JOIN Staff S ON CA.STAFF_ID = S.STAFF_ID
WHERE S.Status = 'Active';
\end{lstlisting}

\subsection*{2. Find Staff IDs of staff who are either 'Active' or have issued at least one prescription.}
\noindent\textbf{Relational Algebra:}\\
\begin{align*}
\left( \pi_{\text{STAFF\_ID}}(\sigma_{\text{Status='Active'}}(\text{Staff})) \right) \cup \left( \pi_{\text{STAFF\_ID}}(\text{ClinicalActivity} \Join_{\text{CAID}} \text{Prescription}) \right)
\end{align*}
\noindent\textbf{SQL:}\\
\begin{lstlisting}[caption={Query for active staff or staff who issued prescriptions.}]
SELECT STAFF_ID FROM Staff WHERE Status = 'Active'
UNION
SELECT CA.STAFF_ID
FROM ClinicalActivity CA
JOIN Prescription P ON CA.CAID = P.CAID;
\end{lstlisting}

\subsection*{3. Find Hospital IDs of hospitals located in 'Benguerir' or having at least one department with the specialty 'Cardiology'.}
\noindent\textbf{Relational Algebra:}\\
\[
\left( \pi_{\text{HID}}(\sigma_{\text{City='Benguerir'}}(\text{Hospital})) \right) \cup \left( \pi_{\text{HID}}(\sigma_{\text{Specialty='Cardiology'}}(\text{Department})) \right)
\]
\noindent\textbf{SQL:}\\
\begin{lstlisting}[caption={Query for hospitals in Benguerir or with a Cardiology department.}]
SELECT HID FROM Hospital WHERE City = 'Benguerir'
UNION
SELECT HID FROM Department WHERE Specialty = 'Cardiology';
\end{lstlisting}

\subsection*{4. Find Hospital IDs of hospitals that have both 'Cardiology' and 'Pediatrics' departments.}
\noindent\textbf{Relational Algebra:}\\
\[
\left( \pi_{\text{HID}}(\sigma_{\text{Specialty='Cardiology'}}(\text{Department})) \right) \cap \left( \pi_{\text{HID}}(\sigma_{\text{Specialty='Pediatrics'}}(\text{Department})) \right)
\]
\noindent\textbf{SQL:}\\
\begin{lstlisting}[caption={Query for hospitals with both Cardiology and Pediatrics.}]
SELECT HID FROM Department WHERE Specialty = 'Cardiology'
INTERSECT
SELECT HID FROM Department WHERE Specialty = 'Pediatrics';
\end{lstlisting}

\subsection*{5. Find staff members who have worked in every department of the hospital with HID = 1.}
\noindent\textbf{Relational Algebra:}\\
\[
\left( \pi_{\text{STAFF\_ID, Dep\_ID}}(\text{Work\_in}) \right) / \left( \pi_{\text{DEP\_ID}}(\sigma_{\text{HID}=1}(\text{Department})) \right)
\]
\noindent\textbf{SQL:}\\
\begin{lstlisting}[caption={Query for staff who worked in all departments of Hospital 1.}]
SELECT W.STAFF_ID
FROM Work_in W
WHERE W.Dep_ID IN (SELECT DEP_ID FROM Department WHERE HID = 1)
GROUP BY W.STAFF_ID
HAVING COUNT(DISTINCT W.Dep_ID) = (
    SELECT COUNT(*) FROM Department WHERE HID = 1
);
\end{lstlisting}

\subsection*{6. Find staff members who participated in every clinical activity of the department with DEP\_ID = 2.}
\noindent\textbf{Relational Algebra:}\\
\[
\left( \pi_{\text{STAFF\_ID, CAID}}(\text{ClinicalActivity}) \right) / \left( \pi_{\text{CAID}}(\sigma_{\text{DEP\_ID}=2}(\text{ClinicalActivity})) \right)
\]
\noindent\textbf{SQL:}\\
\begin{lstlisting}[caption={Query for staff who participated in all activities of Department 2.}]
SELECT STAFF_ID
FROM ClinicalActivity
WHERE DEP_ID = 2
GROUP BY STAFF_ID
HAVING COUNT(DISTINCT CAID) = (
    SELECT COUNT(*) FROM ClinicalActivity WHERE DEP_ID = 2
);
\end{lstlisting}

\subsection*{7. Find pairs of staff members (s1, s2) such that s1 has handled more clinical activities than s2.}
\noindent\textbf{Relational Algebra:}\\
Let $\text{StaffCounts} \leftarrow \gamma_{\text{STAFF\_ID; COUNT(CAID)} \rightarrow \text{ActivityCount}}(\text{ClinicalActivity})$
\[
\pi_{\text{s1.STAFF\_ID, s2.STAFF\_ID}} \left( (\rho_{\text{s1}}(\text{StaffCounts})) \Join_{\text{s1.ActivityCount} > \text{s2.ActivityCount}} (\rho_{\text{s2}}(\text{StaffCounts})) \right)
\]
\noindent\textbf{SQL:}\\
\begin{lstlisting}[caption={Query for pairs of staff based on activity count.}]
SELECT s1.STAFF_ID, s2.STAFF_ID
FROM 
    (SELECT STAFF_ID, COUNT(CAID) AS ActivityCount 
     FROM ClinicalActivity GROUP BY STAFF_ID) AS s1,
    (SELECT STAFF_ID, COUNT(CAID) AS ActivityCount 
     FROM ClinicalActivity GROUP BY STAFF_ID) AS s2
WHERE s1.ActivityCount > s2.ActivityCount;
\end{lstlisting}

\subsection*{8. Find Patient IDs of patients who had clinical activities with at least two different staff members.}
\noindent\textbf{Relational Algebra:}\\
\begin{align*}
\pi_{\text{IID}} \Big( 
    \sigma_{\text{StaffCount} \geq 2} \big( 
        \gamma_{\text{IID; COUNT(DISTINCT STAFF\_ID)} \rightarrow \text{StaffCount}} (\text{ClinicalActivity}) 
    \big) 
\Big)
\end{align*}
\noindent\textbf{SQL:}\\
\begin{lstlisting}[caption={Query for patients seen by two or more different staff.}]
SELECT IID
FROM ClinicalActivity
GROUP BY IID
HAVING COUNT(DISTINCT STAFF_ID) >= 2;
\end{lstlisting}

\subsection*{9. Find CAIDs of clinical activities performed in September 2025 at hospitals located in "Benguerir".}
\noindent\textbf{Relational Algebra:}\\
\begin{align*}
\pi_{\text{CAID}} \Big( \sigma_{\substack{\text{City='Benguerir'} \\ \land \text{Date} \geq \text{'2025-09-01'} \\ \land \text{Date} \leq \text{'2025-09-30'}}} \! \!
    \Big( 
        & \text{ClinicalActivity} \\
        & \Join_{\text{ClinicalActivity.DEP\_ID} = \text{Department.DEP\_ID}} \text{Department} \\
        & \Join_{\text{Department.HID} = \text{Hospital.HID}} \text{Hospital}
    \Big)
\Big)
\end{align*}
\noindent\textbf{SQL:}\\
\begin{lstlisting}[caption={Query for activities in Benguerir during September 2025.}]
SELECT CA.CAID
FROM ClinicalActivity CA
JOIN Department D ON CA.DEP_ID = D.DEP_ID
JOIN Hospital H ON D.HID = H.HID
WHERE H.City = 'Benguerir'
  AND CA.Date BETWEEN '2025-09-01' AND '2025-09-30';
\end{lstlisting}

\subsection*{10. Find Staff IDs of staff who have issued more than one prescription.}
\noindent\textbf{Relational Algebra:}\\
\begin{align*}
\pi_{\text{STAFF\_ID}} \Big( 
    \sigma_{\text{PrescriptionCount} > 1} \big( 
        \gamma_{\text{STAFF\_ID; COUNT(DISTINCT PID)} \rightarrow \text{PrescriptionCount}} \\ (\text{ClinicalActivity} \Join_{\text{ClinicalActivity.CAID} = \text{Prescription.CAID}} \text{Prescription}) 
    \big) 
\Big)
\end{align*}
\noindent\textbf{SQL:}\\
\begin{lstlisting}[caption={Query for staff who have issued more than one prescription.}]
SELECT CA.STAFF_ID
FROM ClinicalActivity CA
JOIN Prescription P ON CA.CAID = P.CAID
GROUP BY CA.STAFF_ID
HAVING COUNT(DISTINCT P.PID) > 1;
\end{lstlisting}

\subsection*{11. List IIDs of patients who have scheduled appointments in more than one department.}
\noindent\textbf{Relational Algebra:}\\
Let $\text{Scheduled} \leftarrow \sigma_{\text{Status='Scheduled'}}(\text{Appointment})$
\begin{align*}
\pi_{\text{IID}} \Big( 
    \sigma_{\text{DeptCount} > 1} \big( 
        \gamma_{\text{IID; COUNT(DISTINCT DEP\_ID)} \rightarrow \text{DeptCount}} (\\ \text{Scheduled} \Join_{\text{Scheduled.CAID} = \text{ClinicalActivity.CAID}} \text{ClinicalActivity}) 
    \big) 
\Big)
\end{align*}
\noindent\textbf{SQL:}\\
\begin{lstlisting}[caption={Query for patients with appointments in multiple departments.}]
SELECT CA.IID
FROM ClinicalActivity CA
JOIN Appointment A ON CA.CAID = A.CAID
WHERE A.Status = 'Scheduled'
GROUP BY CA.IID
HAVING COUNT(DISTINCT CA.DEP_ID) > 1;
\end{lstlisting}

\subsection*{12. Find Staff IDs who have no scheduled appointments on the day of the Green March holiday (November 6).}
\noindent\textbf{Relational Algebra:}\\
\begin{align*}
\Big( \pi_{\text{STAFF\_ID}}(\text{Staff}) \Big) - \Big( \pi_{\text{STAFF\_ID}} \big( 
    \sigma_{\substack{\text{Status='Scheduled'} \\ \land \text{EXTRACT(MONTH FROM Date)=11} \\ \land \text{EXTRACT(DAY FROM Date)=6}}}
    (\\ \text{ClinicalActivity} \Join_{\text{ClinicalActivity.CAID} = \text{Appointment.CAID}} \text{Appointment}) \big) 
\Big)
\end{align*}
\noindent\textbf{SQL:}\\
\begin{lstlisting}[caption={Query for staff with no appointments on November 6.}]
SELECT STAFF_ID FROM Staff
EXCEPT
SELECT CA.STAFF_ID
FROM ClinicalActivity CA
JOIN Appointment A ON CA.CAID = A.CAID
WHERE A.Status = 'Scheduled'
  AND EXTRACT(MONTH FROM CA.Date) = 11
  AND EXTRACT(DAY FROM CA.Date) = 6;
\end{lstlisting}

\subsection*{13. Find departments whose average number of clinical activities is below the global departmental average.}
\noindent\textbf{Relational Algebra:}\\
Let $\text{DeptCounts} \leftarrow \gamma_{\text{DEP\_ID; COUNT(CAID)} \rightarrow \text{NumActivities}}(\text{ClinicalActivity})$ \\
Let $\text{GlobalAvg} \leftarrow \gamma_{\text{AVG(NumActivities)} \rightarrow \text{AvgVal}}(\text{DeptCounts})$
\[
\pi_{\text{DEP\_ID}} ( \text{DeptCounts} \Join_{\text{DeptCounts.NumActivities} < \text{GlobalAvg.AvgVal}} \text{GlobalAvg} )
\]
\noindent\textbf{SQL:}\\
\begin{lstlisting}[caption={Query for departments with below-average activity count.}]
SELECT DEP_ID
FROM ClinicalActivity
GROUP BY DEP_ID
HAVING COUNT(CAID) < (
    SELECT AVG(ActivityCount)
    FROM (
        SELECT COUNT(CAID) AS ActivityCount
        FROM ClinicalActivity
        GROUP BY DEP_ID
    ) AS DepartmentActivityCounts
);
\end{lstlisting}

\subsection*{14. For each staff member, return the patient who has the greatest number of completed appointments with that staff member.}
\noindent\textbf{Relational Algebra:}\\
.....
\vspace{1em}

\noindent\textbf{SQL:}\\
\begin{lstlisting}[caption={Query to find the top patient for each staff member.}]
SELECT spc.STAFF_ID, spc.IID
FROM (
    SELECT CA.STAFF_ID, CA.IID, COUNT(*) AS ApptCount
    FROM ClinicalActivity CA
    JOIN Appointment A ON CA.CAID = A.CAID
    WHERE A.Status = 'Completed'
    GROUP BY CA.STAFF_ID, CA.IID
) AS spc
WHERE spc.ApptCount = (
    SELECT MAX(ApptCount)
    FROM (
        SELECT CA.STAFF_ID, CA.IID, COUNT(*) AS ApptCount
        FROM ClinicalActivity CA
        JOIN Appointment A ON CA.CAID = A.CAID
        WHERE A.Status = 'Completed'
        GROUP BY CA.STAFF_ID, CA.IID
    ) AS inner_spc
    WHERE inner_spc.STAFF_ID = spc.STAFF_ID
);
\end{lstlisting}

\subsection*{15. List patients who had at least 3 emergency admissions during the year 2024.}
\noindent\textbf{Relational Algebra:}\\
\begin{align*}
\pi_{\text{IID}} \Big( 
    \sigma_{\text{EmergencyCount} \geq 3}\big( 
        \gamma_{\text{IID; COUNT(CAID)} \rightarrow \text{EmergencyCount}} (
            \sigma_{\text{EXTRACT(YEAR FROM Date)='2024'}} \\ (\text{ClinicalActivity} \Join_{\text{ClinicalActivity.CAID} = \text{Emergency.CAID}} \text{Emergency})
        )
    \big) 
\Big)
\end{align*}
\noindent\textbf{SQL:}\\
\begin{lstlisting}[caption={Query for patients with 3+ emergency admissions in 2024.}]
SELECT CA.IID
FROM ClinicalActivity CA
JOIN Emergency E ON CA.CAID = E.CAID
WHERE EXTRACT(YEAR FROM CA.Date) = 2024
GROUP BY CA.IID
HAVING COUNT(E.CAID) >= 3;
\end{lstlisting}

\newpage
\section{Part 2: Refinement (Functional Dependencies)}
\subsection*{Summary of Functional Dependencies by Relation}
\begin{enumerate}[leftmargin=*, itemsep=1em]
    \item \textbf{Patient}
    \begin{itemize}[leftmargin=*]
        \item $ \text{IID} \rightarrow \{ \text{CIN, FullName, Birth, Sex, BloodGroup, Phone} \} $
        \item $ \text{CIN} \rightarrow \{ \text{IID, FullName, Birth, Sex, BloodGroup, Phone} \} $
        \item \textit{Justification:} IID is the primary key; CIN is a candidate key.
    \end{itemize}

    \item \textbf{Hospital}
    \begin{itemize}[leftmargin=*]
        \item $ \text{HID} \rightarrow \{ \text{Name, City, Region} \} $
        \item \textit{Justification:} HID is the primary key.
    \end{itemize}

    \item \textbf{Department}
    \begin{itemize}[leftmargin=*]
        \item $ \text{DEP\_ID} \rightarrow \{ \text{HID, Name, Specialty} \} $
        \item $ \{\text{HID, Name}\} \rightarrow \{ \text{DEP\_ID, Specialty} \} $
        \item \textit{Justification:} Unique department ID globally, and unique department name within a hospital.
    \end{itemize}

    \item \textbf{Staff}
    \begin{itemize}[leftmargin=*]
        \item $ \text{STAFF\_ID} \rightarrow \{ \text{FullName, Status} \} $
        \item \textit{Justification:} STAFF\_ID is the primary key.
    \end{itemize}

    \item \textbf{Work\_in}
    \begin{itemize}[leftmargin=*]
        \item $ \{\text{STAFF\_ID, Dep\_ID}\} \rightarrow \{ \} $
        \item \textit{Justification:} Pure many-to-many relationship; no non-key attributes.
    \end{itemize}

    \item \textbf{ClinicalActivity}
    \begin{itemize}[leftmargin=*]
        \item $ \text{CAID} \rightarrow \{ \text{IID, STAFF\_ID, DEP\_ID, Date, Time} \} $
        \item \textit{Justification:} CAID is the primary key.
    \end{itemize}

    \item \textbf{Appointment}
    \begin{itemize}[leftmargin=*]
        \item $ \text{CAID} \rightarrow \{ \text{Reason, Status} \} $
        \item \textit{Justification:} CAID is both the primary and foreign key to ClinicalActivity.
    \end{itemize}

    \item \textbf{Emergency}
    \begin{itemize}[leftmargin=*]
        \item $ \text{CAID} \rightarrow \{ \text{TriageLevel, Outcome} \} $
        \item \textit{Justification:} CAID is both the primary and foreign key to ClinicalActivity.
    \end{itemize}

    \item \textbf{Insurance}
    \begin{itemize}[leftmargin=*]
        \item $ \text{InsID} \rightarrow \{ \text{Type} \} $
        \item \textit{Justification:} InsID is the primary key.
    \end{itemize}

    \item \textbf{Expense}
    \begin{itemize}[leftmargin=*]
        \item $ \text{ExpID} \rightarrow \{ \text{InsID, CAID, Total} \} $
        \item $ \text{CAID} \rightarrow \{ \text{ExpID, InsID, Total} \} $
        \item \textit{Justification:} ExpID is the primary key; CAID is unique, establishing a 1-to-1 relationship with ClinicalActivity.
    \end{itemize}

    \item \textbf{Medication}
    \begin{itemize}[leftmargin=*]
        \item $ \text{MID} \rightarrow \{ \text{Name, Form, Strength, ActiveIngredient, TherapeuticClass, Manufacturer} \} $
        \item \textit{Justification:} MID is the primary key.
    \end{itemize}

    \item \textbf{Stock}
    \begin{itemize}[leftmargin=*]
        \item $ \{\text{HID, MID, StockTimestamp}\} \rightarrow \{ \text{UnitPrice, Qty, ReorderLevel} \} $
        \item \textit{Justification:} Composite key (HID, MID, StockTimestamp) represents stock history per hospital-medication pair.
    \end{itemize}

    \item \textbf{Prescription}
    \begin{itemize}[leftmargin=*]
        \item $ \text{PID} \rightarrow \{ \text{CAID, DateIssued} \} $
        \item $ \text{CAID} \rightarrow \{ \text{PID, DateIssued} \} $
        \item \textit{Justification:} PID is the primary key; CAID is unique.
    \end{itemize}
    
    \item \textbf{Includes}
    \begin{itemize}[leftmargin=*]
        \item $ \{\text{PID, MID}\} \rightarrow \{ \text{Dosage, Duration} \} $
        \item \textit{Justification:} Many-to-many relationship between Prescription and Medication.
    \end{itemize}

    \item \textbf{ContactLocation}
    \begin{itemize}[leftmargin=*]
        \item $ \text{CLID} \rightarrow \{ \text{City, Province, Street, Number, PostalCode, Phone\_Location} \} $
        \item Additional Transitive Dependencies:
            \begin{itemize}
                \item $ \{\text{City, Province, Street}\} \rightarrow \text{PostalCode} $
                \item $ \text{PostalCode} \rightarrow \{ \text{City, Province} \} $
                \item $ \{\text{Number, Street, City, Province}\} \rightarrow \text{Phone\_Location} $
            \end{itemize}
        \item \textit{Justification:} CLID is the primary key; transitive dependencies reveal normalization issues.
    \end{itemize}

    \newpage

    \item \textbf{Have}
    \begin{itemize}[leftmargin=*]
        \item $ \{\text{IID, CLID}\} \rightarrow \{ \} $
        \item \textit{Justification:} Many-to-many relationship between Patient and ContactLocation.
    \end{itemize}

    \item \textbf{Covers}
    \begin{itemize}[leftmargin=*]
        \item $ \{\text{InsID, IID}\} \rightarrow \{ \} $
        \item \textit{Justification:} This is a pure many-to-many relationship between Insurance and Patient. The combination (InsID, IID) is the primary key with no non-key attributes.
    \end{itemize}

\end{enumerate}
\end{document}